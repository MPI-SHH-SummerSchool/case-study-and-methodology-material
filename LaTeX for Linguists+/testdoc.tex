%!TEX TS-program = xelatex

\documentclass[12pt]{article}

% this sets nice a4 page or letter page
	\usepackage[a4paper]{geometry} % for the rest of the world
	% \usepackage[letterpaper, margin=1in]{geometry} % for US 

% this is to set up a glossary
	\usepackage{glossaries}

% for adding comments and todo notes (great for collaboration):
	\usepackage{pdfcomment,todonotes}

% this loads the unicode and font-specific packages
	\usepackage{fontspec,xltxtra,xunicode}
%   	 \usepackage{libertine}

% this loads the linguistic stuff
% for doc consult http://mirror.switch.ch/ftp/mirror/tex/macros/latex/contrib/gb4e/gb4e-doc.pdf
	\usepackage{gb4e, cgloss}
% formatting the examples for nicer look
		\let\eachwordone=\itshape % italic data line (default is \rm)
		\let\eachwordtwo=\small % for small print glosses

% and advanced bibliography stuff
	\usepackage{natbib}
		\renewcommand{\bibfont}{\small}
		\renewcommand{\bibsep}{1pt} % space between entries
		\setcitestyle{comma,aysep={},notesep={:}} % comma between citations (for semicolon, type 'semicolon'); no comma between author and year; no space between citation and post-citation note
		\newcommand\cites[1]{\citeauthor{#1}'s\ (\citeyear{#1})} % to use genitive s with citations


% nice hyperrefs, with some pretty stuff
   	\usepackage{hyperref} % hyperlinks
		\hypersetup{colorlinks=true, citecolor=blue, linkcolor=black, urlcolor=black, pdfauthor={}, pdftitle={}, breaklinks=true}


% to tinker with lists
   	\usepackage{enumitem}


% using leipzig glosses
	\usepackage{leipzig}
		\newleipzig{conv}{conv}{converb}


% tells LaTeX to actually make the glossary
	\makeglossaries


% set up the title page
	\title{Template}
% for font styles (and other stuff), see http://en.wikibooks.org/wiki/LaTeX/Formatting#Font_Styles
	\author{Me Me}
	\date{\today} 


\begin{document}


% generate title
\maketitle

% generate a table of contents
\tableofcontents


% this is a section without numbering
\section*{Introduction}
\label{sec:intro} % this is a label

Let's type some diacritics and special characters: á è, Ὁ ɯ ʉ ħ -- they just show up in the pdf, no extra work required!

This is a text.
It   does not
matter how many
spaces there are
between the words - it is still the same paragraph.
You can break a line \\ like this.


To start a new paragraph, you can use two empty lines or the \par command if you want.
\pdfcomment[color=blue]{this is a comment!!!!}

If you to put something into quotations, use ``this''. For a nice quote-like text formatting, you can use the:
\begin{quote}
	quote environment. It nicely wraps the stuff into a paragraph offset by two margins.
\end{quote}


You can also do footnotes\footnote{pretty neat, huh?} easily.


% this is a numbered section
\section{Some part}

This is a reference to section \nameref{sec:intro} on the page \pageref{sec:intro}.

% here's a subsection, also numbered
\subsection{Lists}

Let's list some stuff:

\begin{enumerate}
	\item This is an enumerate item
	\item Another one
	\item \begin{enumerate}
		\item nested
		\item and again
		\end{enumerate}
	\item last one
\end{enumerate}


or itemize:

\begin{itemize}
	\item This is an itemized item
	\item Another one
	\item \begin{itemize}
		\item nested
		\item and again
		\end{itemize}
	\item last one
\end{itemize}


\subsection{Tables and Figures}

Here's a table. Do not worry if it shows up in another spot in the PDF. That is normal, because it's a float. More on that in the workshop.

\begin{table}
	\centering
	\begin{tabular}{|l|r|c|} \hline
		a   & b   & d   \\ \hline
		df  & df  & eg  \\ \hline
		bla & bla & bla \\ \hline
	\end{tabular}
	\caption{Lots of blabla}
	\label{tab:firsttable}
\end{table}
% TODO: \usepackage{graphicx} required



We can of course also insert figures. Like tables they `float' around in the document. 

\begin{figure}
	\centering
	\includegraphics[width=0.5\textwidth]{testpic.jpg}
	\caption{Not the type of latex we are working with}
	\label{fig:testpic}
\end{figure}




\section{Examples with and without glosses}

Very similar to quotes (also consult the gb4e pdf manual - just google):

\begin{exe}
	\ex This is a numbered example
	\ex \label{ex:some} You can also label it
\end{exe}

Labels can be recalled in the same way (like see example \ref{ex:some}).


But what we really want are glossed examples, so here we go.

Nice glossed example with a description:
\begin{exe}
	\ex Bernese German \\
	\gll Bärndütsch isch di beschti sprach! \\
		Bernese.German \Cop.\Tsg{} \Art.\Def.\F{} best.\Conv.\F{} language		\\
	\glt `Bernese German is the best language!'\protect\footnotemark \hfill 
	\label{ex:first}
\end{exe}


same without a description:
\begin{exe}
	\ex \gll Bärndütsch isch di beschti sprach! \\
		Bernese.German \Cop.\Tsg{} \Art.\Def.\F{} best.\F{} language		\\
	\glt `Bernese German is the best language!' \hfill 
	\label{ex:second}
\end{exe}


nested examples:
\begin{exe} \ex
	\begin{xlist}
	\ex \gll Gli isch summer.	\\
		soon \Cop.\Tsg{} summer			\\
	\glt `Soon it's summer.' \hfill \citep[]{}
	\label{ex:third}
	\ex \gll Was isch los?	\\
		what \Cop.\Tsg{} loose			\\
	\glt `What's happening?' \hfill \citep[]{}
	\label{ex:fourth}
	\end{xlist}
\end{exe}

% TODO: also have to do this


This is the second example label \ref{ex:second}.


\section{Summary}

Other fun stuff: formulas (\LaTeX was developed for math, after all). Formulas must be surrounded with dollar signs.

$$
	f(x) = \sum_{i=1}^{n} \frac{x}{i}
$$

The underscore is an index: $a_5$, the hat is an exponent: $b_6$. Both these symbols (as well as dollar signs) cannot be used in normal text.
If you need an underscore or dollar sign use in text, you need to escape it with a backslash like so: \_ or \&


Last but not least: bibliography!

Add citations with parentheses like this  \citep{bickel2007typology} or without like this  \citealt{macaulay1996grammar}.
There are more options! Also read up on the natbib package.



\section{How to get a pdf out of this} % (fold)
\label{sec:pdf}

% write note of this
% TODO: this is a what I still have to do


\begin{enumerate}
	\item run \LaTeX{} twice (yes, it has to be twice, otherwise things won't work out in many cases)
	\item run BibTeX once
	\item run \LaTeX{} again
	\item generate pdf (depends on editor how exactly you get there)
\end{enumerate}


% section how_to_get_a_pdf_out_of_this (end)


Now we let the magic work and create our glossary. 

\printglossary[title=Glosses,type=\leipzigtype, style=block]


\bibliographystyle{unified.linguistic}	
\bibliography{testbib}
	
\end{document}
